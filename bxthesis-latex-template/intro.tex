\section{.NET and the Common Language Runtime}
First released in 2000 as the proprietary .NET Framework, the .NET platform is a complete programming ecosystem developed by Microsoft.
While its primary programming language is C\#, the platform itself is language-agnostic,
being built upon a virtual machine execution environment called the Common Language Runtime, or CLR.
Strictly speaking, the .NET platform refers primarily to the rich standard library and development frameworks bundled with the platform,
whereas the CLR itself is a fully independently-specified runtime.
Originally designed as a competitor to Sun Microsystems' Java Virtual Machine~\cite{20yrsdotnet},
the CLR has since eclipsed the JVM in terms of built-in features, such as first-class support for generic types
(as opposed to Java's strategy of runtime type erasure), direct interfaces to unsafe operations like pointer manipulation,
and high interoperability with native C or C++ libraries.

\subsection{Open specification and implementations}
.NET and the CLR have a long history with open source software and community involvement.
In 2001, Ecma International published the first edition of the ECMA-335 standard, which codified the Common Language Infrastructure (CLI):
a collection of specifications for the language runtime, bytecode instruction set, and type system implemented by the CLR.
(It should be noted that editions of this standard do \textit{not} correspond to specific versions of the .NET distribution or the C\# language.
The last update to the standard was the sixth edition published in 2012, whereas both .NET and C\# have had countless updates since then.)
The only implementation at the time was Microsoft's proprietary .NET Framework for Windows, inspiring Linux programmer Miguel de Icaza
to start the Mono project, an open-source cross-platform implementation of the CLI intended to be compatible with Microsoft's .NET.

For more than a decade, this was the status quo: Microsoft continued to publish .NET as a Windows-only framework, intended for enterprise systems and Windows
application development in C\#, whereas Mono was used as the execution environment for cross-platform and open-source projects that wanted to use C\#,
such as the Unity game engine's scripting system. % TODO: cite
This all changed in 2014, when Microsoft published .NET Core, a fully open-source and cross-platform implementation of the .NET stack.
This included publishing the source code for Microsoft's CLR implementation.
In 2020, Microsoft replaced .NET Framework and .NET Core with the unified .NET 5.0, and since then, the .NET ecosystem has embraced
cross-platform execution and open development.
Microsoft later went on to open-source the entirety of the .NET SDK, including their C\# compiler,
and has invited community participation in the evolution process through the .NET Foundation on GitHub.

Of particular relevance to this exploration is the public source code to Microsoft's CLR implementation~\cite{runtime}.
The runtime is written primarily in C++, although it should be noted that the repository is majority-C\# due to the presence of
library source code that requires special internal calls to the runtime.
In particular, the style of C++ used for the implementation is highly complex: pointers are very frequently cast between types,
the class hierarchy is highly nested, and the boundary between garbage-collected (also called ``managed'') code and native code with
manual memory management is frequently crossed.
Not only is this very difficult to grasp for open source contributors, but the unchecked nature of C++ means that there are many
opportunities for memory safety errors to arise and potentially become exposed as a vulnerability in the runtime itself.
These problems have existed for long enough across the C++ community that other programming languages have been developed
explicitly to solve them.

\section{The Rust language}
In 2006, Mozilla employee Graydon Hoare started development on Rust as a personal project,
with the explicit desire to create a safer and more ergonomic alternative to C++.
Recognizing its potential, Mozilla officially sponsored Rust's development in 2009.
Since then, the language has grown tremendously quickly,
fostering an online community of Rust developers that is also highly involved in the evolution process.
Since its first stable release in 2015, countless companies and large open-source projects have recognized its potential
as a more productive replacement for C++.
Rust has also proved a popular choice for embedded programming environments, and in 2022,
Rust achieved the notable accomplishment of being accepted into the Linux kernel.

\subsection{Notable features}
The most distinctive feature of Rust is its memory safety system.
Rust was designed from the ground up to prevent common errors such as null pointer dereferences, use-after-frees, and data races,
which it accomplishes with its famous ownership system, borrow checker, and lifetime syntax.
These are core to the language and so robust that any code outside an \texttt{unsafe} block that compiles
is guaranteed to be free of memory unsafety, concurrency errors (other than deadlocks), or undefined behavior.
Furthermore, these semantic guarantees have been mathematically formalized and proven to be sound~\cite{rustbelt}.

Rust is also famous for its extremely expressive type system, which is inspired by those from functional programming languages such as Scala.
It replaces C++'s textual template programming with true quantified type generics and its complex inheritance system with simple
compositional traits (otherwise known as typeclasses).
When combined with true algebraic data types and pattern matching, this allows for incredible levels of flexibility and abstraction
for both library authors and consumers.

These combine to create Rust's most compelling feature: \textit{zero-cost abstractions}.
Plenty of programming languages with both true memory safety and highly complex type systems exist, but these are often implemented with
garbage collectors, heavyweight runtime systems, or otherwise computationally-intensive bookkeeping, which make such languages unsuitable
for performance-sensitive applications; however, tracking memory ownership at compile-time means that Rust does not need any sort of garbage collection.
Additionally, the highly-optimizing compiler can easily remove dead code and provide optimizations specific to a particular generic instantiation,
meaning that generic programming in Rust comes with no runtime cost.
This is all to say that Rust enables one to get ``close to the metal'' just like C++,
but without sacrificing any of the safety or productivity that a more high-level language could provide.

Finally, Rust's focus on developer productivity is most visible in its build system and tooling.
The compiler is known for its helpful error messages that guide the user directly to the problem, and the official documentation provided is stellar.
Furthermore, dependency management is made extremely easy by the Cargo build tools,
standing in stark contrast to the `CMake hell' of C++.
Developers consistently rate Rust as their most loved/admired programming language~\cite{sosurvey},
which can partially be attributed to this intense focus on developer experience.

\section{Objectives}
This project aims to explore the benefits and drawbacks of using Rust as the implementation language for large-scale
programming language runtimes such as .NET.
In particular, Rust's semantic guarantees have the potential to significantly streamline development within the
complex environment of a language runtime.
However, Rust is not a magic bullet.
These guarantees can sometimes come at the cost of additional complexity: the ownership system encourages one to design programs
based around well-defined object ownership graphs, which can be hard to retrofit existing architectures into, and lifetime tracking
sometimes requires syntactic annotations that add to general code noisiness.
As such, a secondary goal of the project is to examine whether the additional annotations and developer effort
required by Rust constitute a significant burden for language system designers.
In all, this establishes a framework for an analysis of Rust's benefits and trade-offs in the specific realm of large-scale programming language systems.
