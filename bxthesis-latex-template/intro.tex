\section{.NET and the Common Language Runtime}
First released in 2000 as the proprietary .NET Framework, the .NET platform is a complete programming ecosystem developed by Microsoft,
famous for its extremely high popularity in business and industry.
While its primary programming language is famously C\#, the platform itself is language-agnostic,
being built upon a virtual machine execution environment called the Common Language Runtime, or CLR.
Strictly speaking, the .NET platform refers primarily to the rich standard library and development frameworks bundled with the platform,
whereas the CLR itself is a fully independently-specified runtime.
Originally designed as a competitor to Sun Microsystems' Java Virtual Machine~\cite{20yrsdotnet},
the CLR has since eclipsed the JVM in terms of built-in features, such as first-class support for generic types
(as opposed to Java's strategy of runtime type erasure), direct interfaces to unsafe operations like pointer manipulation,
and high interoperability with native C or C++ libraries.

\subsection{Open specification and implementations}
.NET and the CLR have a long history with open source software and community involvement.
In 2001, Ecma International published the first edition of the ECMA-335 standard, which codified the Common Language Infrastructure (CLI):
a collection of specifications for the language runtime, bytecode instruction set, and type system implemented by the CLR.
(It should be noted that editions of this standard do \textit{not} correspond to specific versions of the .NET distribution or the C\# language.
The last update to the standard was the sixth edition published in 2012, whereas both .NET and C\# have had countless updates since then.)
However, the only implementation at the time was Microsoft's proprietary .NET Framework for Windows, inspiring Linux programmer Miguel de Icaza
to start the Mono project, an open-source cross-platform implementation of the CLI intended to be compatible with Microsoft's .NET.

For more than a decade, this was the status quo: Microsoft continued to publish .NET as a Windows-only framework, intended for enterprise systems and Windows
application development in C\#, whereas Mono was used as the execution environment for cross-platform and open-source projects that wanted to use C\#,
such as the Unity game engine's scripting system. % TODO: cite
This all changed in 2014, when Microsoft published .NET Core, a fully open-source and cross-platform implementation of the .NET stack.
This included publishing the source code for Microsoft's CLR implementation.
In 2020, Microsoft replaced .NET Framework and .NET Core with the unified .NET 5.0, and since then, the .NET ecosystem has embraced
cross-platform execution and open development.
Microsoft later went on to open-source the entirety of the .NET SDK, including their C\# compiler,
and has invited community participation in the evolution process through the .NET Foundation on GitHub.

% TODO: relevance?


\section{The Rust language}

\section{Execution of CIL code}
